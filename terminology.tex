% !TEX root = ./paper.tex

\section{Key Terms}

\begin{descripion}

  \item \textbf{Potential of mean force (PMF)} is formally the ``potential'' derived by integrating the mean force along a path or coordinate \cite{Kirkwood-1935}, analogous to the probability density along that coordinate.  Often used interchangeably with \textbf{Free energy curve}, but strictly the PMF does not contain entropic contributions from the Jacobian of the chosen coordinate.

  \item \textbf{Free energy curve} is the free energy as a function of a
  particular coordinate or coordinates.  It is directly related to the
  probability distribution along that coordinate, and implicitly includes the
  entropic contributions from the Jacobian.


  \item \textbf{Collective variable:} any variable composed of displacements of
  multiple particle coordinates.  It could be as simple as the distance between
  two particles, or complex like the fraction of native contacts present in a
  protein folding simulation.

  \item \textbf{Order parameter:}  a collective variable that distinguishes between two or more stable states

  \item \textbf{Reaction coordinate:} 

\end{descripion}
